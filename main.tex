\documentclass[journal,12pt,twocolumn]{IEEEtran}

\usepackage{setspace}
\usepackage{gensymb}

\singlespacing


\usepackage[cmex10]{amsmath}
\usepackage{amsthm}

\usepackage{mathrsfs}
\usepackage{txfonts}
\usepackage{stfloats}
\usepackage{bm}
\usepackage{cite}
\usepackage{cases}
\usepackage{subfig}

\usepackage{longtable}
\usepackage{multirow}

\usepackage{enumitem}
\usepackage{mathtools}
\usepackage{steinmetz}
\usepackage{tikz}
\usepackage{circuitikz}
\usepackage{verbatim}
\usepackage{tfrupee}
\usepackage[breaklinks=true]{hyperref}
\usepackage{graphicx}
\usepackage{tkz-euclide}
\usepackage{float}

\usetikzlibrary{calc,math}
\usepackage{listings}
    \usepackage{color}                                            %%
    \usepackage{array}                                            %%
    \usepackage{longtable}                                        %%
    \usepackage{calc}                                             %%
    \usepackage{multirow}                                         %%
    \usepackage{hhline}                                           %%
    \usepackage{ifthen}                                           %%
    \usepackage{lscape}     
\usepackage{multicol}
\usepackage{chngcntr}

\DeclareMathOperator*{\Res}{Res}

\renewcommand\thesection{\arabic{section}}
\renewcommand\thesubsection{\thesection.\arabic{subsection}}
\renewcommand\thesubsubsection{\thesubsection.\arabic{subsubsection}}

\renewcommand\thesectiondis{\arabic{section}}
\renewcommand\thesubsectiondis{\thesectiondis.\arabic{subsection}}
\renewcommand\thesubsubsectiondis{\thesubsectiondis.\arabic{subsubsection}}


\hyphenation{op-tical net-works semi-conduc-tor}
\def\inputGnumericTable{}                                 %%
\lstset{
%language=C,
frame=single, 
breaklines=true,
columns=fullflexible
}
\begin{document}
\newtheorem{theorem}{Theorem}[section]
\newtheorem{problem}{Problem}
\newtheorem{proposition}{Proposition}[section]
\newtheorem{lemma}{Lemma}[section]
\newtheorem{corollary}[theorem]{Corollary}
\newtheorem{example}{Example}[section]
\newtheorem{definition}[problem]{Definition}

\newcommand{\BEQA}{\begin{eqnarray}}
\newcommand{\EEQA}{\end{eqnarray}}
\newcommand{\define}{\stackrel{\triangle}{=}}
\bibliographystyle{IEEEtran}
\providecommand{\mbf}{\mathbf}
\providecommand{\pr}[1]{\ensuremath{\Pr\left(#1\right)}}
\providecommand{\qfunc}[1]{\ensuremath{Q\left(#1\right)}}
\providecommand{\sbrak}[1]{\ensuremath{{}\left[#1\right]}}
\providecommand{\lsbrak}[1]{\ensuremath{{}\left[#1\right.}}
\providecommand{\rsbrak}[1]{\ensuremath{{}\left.#1\right]}}
\providecommand{\brak}[1]{\ensuremath{\left(#1\right)}}
\providecommand{\lbrak}[1]{\ensuremath{\left(#1\right.}}
\providecommand{\rbrak}[1]{\ensuremath{\left.#1\right)}}
\providecommand{\cbrak}[1]{\ensuremath{\left\{#1\right\}}}
\providecommand{\lcbrak}[1]{\ensuremath{\left\{#1\right.}}
\providecommand{\rcbrak}[1]{\ensuremath{\left.#1\right\}}}
\theoremstyle{remark}
\newtheorem{rem}{Remark}
\newcommand{\sgn}{\mathop{\mathrm{sgn}}}
\providecommand{\abs}[1]{\left\vert#1\right\vert}
\providecommand{\res}[1]{\Res\displaylimits_{#1}} 
\providecommand{\norm}[1]{\left\lVert#1\right\rVert}
%\providecommand{\norm}[1]{\lVert#1\rVert}
\providecommand{\mtx}[1]{\mathbf{#1}}
\providecommand{\mean}[1]{E\left[ #1 \right]}
\providecommand{\fourier}{\overset{\mathcal{F}}{ \rightleftharpoons}}
%\providecommand{\hilbert}{\overset{\mathcal{H}}{ \rightleftharpoons}}
\providecommand{\system}{\overset{\mathcal{H}}{ \longleftrightarrow}}
	%\newcommand{\solution}[2]{\textbf{Solution:}{#1}}
\newcommand{\solution}{\noindent \textbf{Solution: }}
\newcommand{\cosec}{\,\text{cosec}\,}
\providecommand{\dec}[2]{\ensuremath{\overset{#1}{\underset{#2}{\gtrless}}}}
\newcommand{\myvec}[1]{\ensuremath{\begin{pmatrix}#1\end{pmatrix}}}
\newcommand{\mydet}[1]{\ensuremath{\begin{vmatrix}#1\end{vmatrix}}}
\numberwithin{equation}{subsection}
\makeatletter
\@addtoreset{figure}{problem}
\makeatother
\let\StandardTheFigure\thefigure
\let\vec\mathbf
\renewcommand{\thefigure}{\theproblem}
\def\putbox#1#2#3{\makebox[0in][l]{\makebox[#1][l]{}\raisebox{\baselineskip}[0in][0in]{\raisebox{#2}[0in][0in]{#3}}}}
     \def\rightbox#1{\makebox[0in][r]{#1}}
     \def\centbox#1{\makebox[0in]{#1}}
     \def\topbox#1{\raisebox{-\baselineskip}[0in][0in]{#1}}
     \def\midbox#1{\raisebox{-0.5\baselineskip}[0in][0in]{#1}}
\vspace{3cm}
\title{ASSIGNMENT 8}
\author{SOWMYA BANDI}
\maketitle
\newpage
\bigskip
\renewcommand{\thefigure}{\theenumi}
\renewcommand{\thetable}{\theenumi}
Download all python codes from 
%
\begin{lstlisting}
https://github.com/Sowmyabandi99/Assignment8/blob/main/assignment8.py
\end{lstlisting}
%
and latex-tikz codes from 
%
\begin{lstlisting}
https://github.com/Sowmyabandi99/Assignment8/blob/main/main.tex
\end{lstlisting}
%
\section{Question No 2.50}
Balance the following chemical equation.
\begin{align}
NaOH + H_2SO_4 \xrightarrow{} Na_2SO_4 + H_2O \label{eq1}
\end{align}
%
\section{SOLUTION} 
Let the balanced version of \eqref{eq1} be
\begin{align}
   x_{1}NaOH + x_{2}H_2SO_4 \xrightarrow{} 
   x_{3}Na_2SO_4 + x_{4}H_2O \label{eq2}
\end{align}

which results in the following equations:
\begin{align}
    (x_{1}-2x_{3}) Na= 0\\
    (x_{1}+4x_{2}-4x_{3}-x_{4}) O= 0\\
    (x_{1}+2x_{2}-2x_{4}) H=0\\
    (x_{2}-x_{3}) S= 0
\end{align}

which can be expressed as
\begin{align}
    x_{1}+ 0.x_{2}- 2x_{3}+ 0.x_{4} = 0\\
    x_{1}+ 4x_{2}- 4x_{3}- x_{4} = 0\\
    x_{1}+ 2x_{2}+ 0.x_{3}- 2x_{4} = 0\\
    0.x_{1}+ x_{2}- x_{3}+ 0.x_{4} = 0
\end{align}

resulting in the matrix equation
\begin{align}
    \myvec{1 & 0 & -2 & 0\\
           1 & 4 & -4 & -1\\
           1 & 2 & 0 & -2\\
           0 & 1 & -1 & 0}\vec{x}
           =\vec{0}    \label{eq3}
\end{align}

where,
\begin{align}
   \vec{x}= \myvec{x_{1}\\x_{2}\\x_{3}\\x_{4}}
\end{align}

\eqref{eq3} can be reduced as follows:
\begin{align}
    \myvec{1 & 0 & -2 & 0\\
           1 & 4 & -4 & -1\\
           1 & 2 & 0 & -2\\
           0 & 1 & -1 & 0}
    \xleftrightarrow[R_{3}\leftarrow R_3-R_{1}]{R_{2}\leftarrow R_2- R_1}
    \myvec{1 & 0 & -2 & 0\\
           0 & 4 & -2 & -1\\
           0 & 2 & 2 & -2\\
           0 & 1 & -1 & 0}\\
    \xleftrightarrow{R_2 \leftarrow \frac{R_2}{4}}
    \myvec{1 & 0 & -2 & 0\\
          0 & 1 & -\frac{1}{2} & -\frac{1}{4}\\
          0 & 2 & 2 & -2\\
          0 & 1 & -1 & 0}\\
    \xleftrightarrow[R_4 \leftarrow R_4 - R_2]{R_3 \leftarrow R_3 - 2R_2}
    \myvec{1 & 0 & -2 & 0\\
           0 & 1 & -\frac{1}{2} & -\frac{1}{4}\\
           0 & 0 & 3 & -\frac{3}{2}\\
           0 & 0 & -\frac{1}{2} & \frac{1}{4}}\\
    \xleftrightarrow{R_3 \leftarrow \frac{R_3}{3}}
    \myvec{1 & 0 & -2 & 0\\
           0 & 1 & -\frac{1}{2} & -\frac{1}{4}\\
           0 & 0 & 1 & -\frac{1}{2}\\
           0 & 0 & -\frac{1}{2} & \frac{1}{4}}\\
    \xleftrightarrow[R_{4}\leftarrow R_4+\frac{R_3}{2}]{R_{2}\leftarrow R_2+ \frac{R_3}{2}}
    \myvec{1 & 0 & -2 & 0\\
           0 & 1 & 0 & -\frac{1}{2}\\
           0 & 0 & 1 & -\frac{1}{2}\\
           0 & 0 & 0 & 0}\\
    \xleftrightarrow{R_1 \leftarrow R_1+2R_3}
    \myvec{1 & 0 & 0 & -1\\
           0 & 1 & 0 & -\frac{1}{2}\\
           0 & 0 & 1 & -\frac{1}{2}\\
           0 & 0 & 0 & 0}
\end{align}

Thus,
\begin{align}
    x_1=x_4, x_2= \frac{1}{2}x_4, x_3=\frac{1}{2}x_4\\
    \implies \quad\vec{x}= x_4\myvec{1\\ \frac{1}{2}\\ \frac{1}{2}\\1} 
\end{align} 
by substituting $x_4= 2$
\begin{align}
    \vec{x}=\myvec{2\\1\\1\\2}
\end{align}

Hence, \eqref{eq2} finally becomes
\hfill\break
%\vspace{5mm}  finally becomes
\begin{align}
    2 NaOH + H_2SO_4 \xrightarrow{} 
    Na_2SO_4 + 2 H_2O
\end{align}
\end{document}